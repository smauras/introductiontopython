\documentclass[
    aspectratio=169, 
    usepdftitle=false, 
    xcolor={dvipsnames},
    hyperref={
        colorlinks,
        linkcolor=black,
        urlcolor=blue}
    ]{beamer}
\usetheme{Madrid}
\usepackage{graphicx}
\usepackage{listings}
\usepackage{soul}
\usepackage{amsmath}
\lstset{
    numbers=left,
    xleftmargin=2em,
    frame=single,
    framexleftmargin=0em,
    basicstyle=\footnotesize\ttfamily,
    xleftmargin=.075\textwidth,
    xrightmargin=.075\textwidth,
    numberstyle=\footnotesize\ttfamily,
    upquote=true,
    framesep=10pt,
    numbersep=20pt,
    keywordstyle=\bfseries,
    stringstyle=\textit,
    showstringspaces=false,
    columns=fixed,
    breaklines=true,
}

\lstdefinestyle{output}{
    numbers=none,
    xleftmargin=2em,
    frame=single,
    framexleftmargin=0em,
    basicstyle=\footnotesize\ttfamily,
    xleftmargin=.075\textwidth,
    xrightmargin=.075\textwidth,
    numberstyle=\footnotesize\ttfamily,
    upquote=true,
    framesep=10pt,
    numbersep=20pt,
    keywordstyle=\bfseries,
    stringstyle=\textit,
    showstringspaces=false,
    columns=fixed,
    breaklines=true,
    backgroundcolor=\color{lightgray},
}
% \usepackage{xcolor}

\hypersetup{colorlinks,urlcolor=blue}
\addtobeamertemplate{headline}{\hypersetup{linkcolor=.}}{}
\addtobeamertemplate{footline}{\hypersetup{linkcolor=.}}{}

\definecolor{Light}{gray}{.90}
\sethlcolor{Light}

\let\OldTexttt\texttt
\renewcommand{\texttt}[1]{\OldTexttt{\hl{#1}}}% will affect all \texttt

\title[Introduction to Python]{Introduction to Python}
\subtitle{Lecture 3: Functions and Classes}
\author{Daniel Kadyrov}
\date{July 4th, 2023}

\begin{document}

\begin{frame}
    \titlepage
\end{frame}

\begin{frame}{Functions}
    \begin{itemize}
        \item A function is a block of code which only runs when it is called.
        \item You can pass data, known as parameters, into a function.
        \item A function can return data as a result.
        \item Functions are used to perform certain actions, and they are important for reusing code
        \item Functions are defined using the \texttt{def} keyword.
        \item The \texttt{def} keyword is followed by the function name, parentheses \texttt{()}, and a colon \texttt{:}.
        \item The \texttt{return} keyword is used to return a value from the function.
    \end{itemize}
\end{frame}

\begin{frame}[fragile]{Functions}
    The following code snippet defines a function that prints a string passed to it as an argument in reverse order: 

    \begin{lstlisting}[language=Python]
def reverse_string(string):
    print(string[::-1])
    \end{lstlisting}

    The function can be called as follows:

    \begin{lstlisting}[language=Python]
reverse_string("Hello World!")
    \end{lstlisting}

    The output of the function call is:

    \begin{lstlisting}[language=Python]
!dlroW olleH
    \end{lstlisting}
\end{frame}

\begin{frame}[fragile]{Functions}
    The following code snippet defines a function that splits a sentence into a list of words, capitalizes each word in the list, and returns the capitalized sentence:

    \begin{lstlisting}[language=Python]
def capitalize_sentence(sentence):
    words = sentence.split()
    capitalized_words = []
    for word in words:
        capitalized_words.append(word.capitalize())
    return " ".join(capitalized_words)
    \end{lstlisting}

    The function can be called as follows:

    \begin{lstlisting}[language=Python]
sentence = "Code is like humor. When you have to explain it, it's bad."
new_sentence = capitalize_sentence(sentence)
    \end{lstlisting}

    The output of the function call is:

    \begin{lstlisting}[language=Python]
Code Is Like Humor. When You Have To Explain It, It's Bad.
    \end{lstlisting}
\end{frame}

\begin{frame}[fragile]{Functions}
    \framesubtitle{Recursive Functions}
    A recursive function is a function that calls itself during its execution. This enables the function to repeat itself several times, outputting the result and the end of each iteration.\\~\



\end{frame}

\begin{frame}[fragile]{Functions}
    \framesubtitle{Recursive Functions}

    The following code snippet defines a recursive function that calculates the factorial of a number. Mathematically, a factorial is expressed the following way: 

    \begin{equation*}
        n! = n \times (n-1) \times (n-2) \times \dots \times 1
    \end{equation*}

    This can be expressed in Python as follows:

    \begin{lstlisting}[language=Python]
def factorial(n):
    if n == 1:
        return 1
    else:
        return n * factorial(n-1)
    \end{lstlisting}

    The function can be called as follows:

    \begin{lstlisting}[language=Python]
factorial(5)
    \end{lstlisting}

    The output of the function call is: \texttt{120}
\end{frame}

\begin{frame}[fragile]{Functions}
    \framesubtitle{Lambda Functions}
    A lambda function is a small anonymous function. It can take any number of arguments, but can only have one expression.\\~\

    The following code snippet defines a lambda function that takes a number as an argument and returns the square of that number:

    \begin{lstlisting}[language=Python]
square = lambda x: x**2
    \end{lstlisting}

    The function can be called as follows:

    \begin{lstlisting}[language=Python]
square(5)
    \end{lstlisting}

    The output of the function call is: \texttt{25}
\end{frame}

\begin{frame}[fragile]{Objects, Classes, and Methods}
    \begin{itemize}
        \item Python is an object-oriented programming language.
        \item Almost everything in Python is an object, with its properties and methods.
        \item A Class is like an object constructor, or a "blueprint" for creating objects.
        \item A Class is defined using the \texttt{class} keyword.
        \item The \texttt{class} keyword is followed by the class name, parentheses \texttt{()}, and a colon \texttt{:}.
        \item The \texttt{self} parameter is a reference to the current instance of the class, and is used to access variables that belong to the class.
        \item It does not have to be named \texttt{self}, you can call it whatever you like, but it has to be the first parameter of any function in the class.
    \end{itemize}
\end{frame}
\end{document}