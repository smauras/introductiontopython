\documentclass[
    aspectratio=169, 
    usepdftitle=false, 
    xcolor={dvipsnames},
    hyperref={
        colorlinks,
        linkcolor=black,
        urlcolor=blue}
    ]{beamer}
\usetheme{Madrid}
\usepackage{graphicx}
\usepackage{listings}
\usepackage{soul}
\usepackage{amsmath}
\lstset{
    numbers=left,
    xleftmargin=2em,
    frame=single,
    framexleftmargin=0em,
    basicstyle=\footnotesize\ttfamily,
    xleftmargin=.075\textwidth,
    xrightmargin=.075\textwidth,
    numberstyle=\footnotesize\ttfamily,
    upquote=true,
    framesep=10pt,
    numbersep=20pt,
    keywordstyle=\bfseries,
    stringstyle=\textit,
    showstringspaces=false,
    columns=fixed,
    breaklines=true,
}

\lstdefinestyle{output}{
    numbers=none,
    xleftmargin=2em,
    frame=single,
    framexleftmargin=0em,
    basicstyle=\footnotesize\ttfamily,
    xleftmargin=.075\textwidth,
    xrightmargin=.075\textwidth,
    numberstyle=\footnotesize\ttfamily,
    upquote=true,
    framesep=10pt,
    numbersep=20pt,
    keywordstyle=\bfseries,
    stringstyle=\textit,
    showstringspaces=false,
    columns=fixed,
    breaklines=true,
    backgroundcolor=\color{lightgray},
}
% \usepackage{xcolor}

\hypersetup{colorlinks,urlcolor=blue}
\addtobeamertemplate{headline}{\hypersetup{linkcolor=.}}{}
\addtobeamertemplate{footline}{\hypersetup{linkcolor=.}}{}

\definecolor{Light}{gray}{.90}
\sethlcolor{Light}

\let\OldTexttt\texttt
\renewcommand{\texttt}[1]{\OldTexttt{\hl{#1}}}% will affect all \texttt

\title[Introduction to Python]{Introduction to Python}
\subtitle{Lecture 5: Classes and Objects}
\author{Daniel Kadyrov}
\date{July 10th, 2023}

\begin{document}

\begin{frame}
    \titlepage
\end{frame}

\begin{frame}[fragile]{Objects and Classes}
    \begin{itemize}
        \item Python is an object-oriented programming language.
        \item Almost everything in Python is an object, with its properties and methods.
        \item A Class is like an object constructor, or a "blueprint" for creating objects.
        \item A Class is defined using the \texttt{class} keyword.
        \item The \texttt{class} keyword is followed by the class name, parentheses \texttt{()}, and a colon \texttt{:}.
        \item The \texttt{self} parameter is a reference to the current instance of the class, and is used to access variables that belong to the class.
        \item It does not have to be named \texttt{self}, you can call it whatever you like, but it has to be the first parameter of any function in the class.
    \end{itemize}
\end{frame}

\begin{frame}[fragile]{Objects and Classes}
    \begin{itemize}
        \item Use the \texttt{init()} function to assign values to object properties, or other operations that are necessary to do when the object is being created.
        \item The \texttt{init()} function is called automatically every time the class is being used to create a new object.
        \item All classes have a function called \texttt{init()}, which is always executed when the class is being initiated.
        \item Use the \texttt{self} parameter to refer to the current instance of the class, and access variables that belongs to the class.
        \item It does not have to be named \texttt{self}, you can call it whatever you like, but it has to be the first parameter of any function in the class.
    \end{itemize}
\end{frame}

\begin{frame}[fragile]{Objects and Classes}
    \framesubtitle{Example}
    \begin{lstlisting}[language=Python]
class MyClass:
    x = 5
    def __init__(self, name, age):
        self.name = name
        self.age = age
    def myfunc(self):
        print("Hello my name is " + self.name)
    def myfunc2(self):
        print("Hello my age is " + str(self.age))
    \end{lstlisting}
\end{frame}
\end{document}