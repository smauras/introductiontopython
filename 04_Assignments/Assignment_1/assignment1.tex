\documentclass{homework}

\Title{Assignment 1}
\DueDate{July 4th, 2023}
\ClassName{Introduction to Programming with Python}
\ClassNumber{Columbia University Pre-College Program}
\ClassSection{Summer Session I}
\Instructor{Instructor: Daniel Kadyrov}
\Author{}

\begin{document}

\maketitle

In this assignment, you will practice using variables, floats, integers, strings, lists, and dictionary methods in Python. You will write a program that performs some simple calculations and manipulations on different types of data. You will submit your program as a .py file on the course website. The file should be named \texttt{LastName\_FirstName\_Assignment1.py} where last name and first name are replaced with your last name and first name.\\

\begin{problem}[1 - Variables and Data Types]

Create four variables and assign them values of different data types: a float, an integer, a string, and a list. Print the values and types of each variable using the \texttt{print} and \texttt{type} functions. For example:

\begin{lstlisting}[language=Python]
x = 3.14
print(x)
print(type(x))
\end{lstlisting}

This will output:

\begin{lstlisting}[style=output]
3.14
<class 'float'>
\end{lstlisting}

\end{problem}

\begin{problem}[2 - Arithmetic Operations]

Using the variables you created in Task 1, perform some arithmetic operations on them and print the results. For example, if you have a variable \texttt{y} that is an integer, you can do:

\begin{lstlisting}[language=Python]
z = x + y
print(z)
print(type(z))
\end{lstlisting}

This will output:

\begin{lstlisting}[style=output]
6.140000000000001
<class 'float'>
\end{lstlisting}

Note that the result is a float, even though one of the operands was an integer. This is because Python automatically converts integers to floats when performing arithmetic operations with floats.

Try different operations such as subtraction, multiplication, division, exponentiation, and modulo. What happens when you try to perform an operation on incompatible data types, such as a string and a list?

\end{problem}

\newpage
\begin{problem}[3 - String Methods]

Using the variable that is a string, apply some string methods on it and print the results. For example, if you have a variable \texttt{s} that is a string, you can do:

\begin{lstlisting}[language=Python]
s = "hello"
s_upper = s.upper()
print(s_upper)
\end{lstlisting}

This will output:

\begin{lstlisting}[style=output]
HELLO
\end{lstlisting}

Some other string methods you can try are \texttt{lower}, \texttt{capitalize}, \texttt{title}, \texttt{strip}, \texttt{replace}, \texttt{split}, \texttt{join}, \texttt{find}, \texttt{count}, \texttt{len}, etc. You can find more information about string methods here:\\

\url{https://docs.python.org/3/library/stdtypes.html#string-methods}

\end{problem}

\begin{problem}[4 - List Methods]

Using the variable that is a list, apply some list methods on it and print the results. For example, if you have a variable \texttt{l} that is a list, you can do:

\begin{lstlisting}[language=Python]
l = [1, 2, 3, 4]
l.append(5)
print(l)
\end{lstlisting}

This will output:

\begin{lstlisting}[style=output]
[1, 2, 3, 4, 5]
\end{lstlisting}

Some other list methods you can try are \texttt{insert}, \texttt{remove}, \texttt{pop}, \texttt{index}, \texttt{sort}, \texttt{reverse}, \texttt{copy}, etc. You can find more information about list methods here:\\

\url{https://docs.python.org/3/tutorial/datastructures.html#more-on-lists}
\end{problem}

\newpage
\begin{problem}[5 -  Dictionary Methods]

Create a dictionary that maps some keys to some values. The keys and values can be of any data type. Print the dictionary using the \texttt{print} function. For example:

\begin{lstlisting}[language=Python]
d = {
    "name": "Alice", 
    "age": 25, 
    "hobbies": ["reading", "writing", "coding"]
}
print(d)
\end{lstlisting}

This will output:

\begin{lstlisting}[style=output]
{
    'name': 'Alice', 
    'age': 25, 
    'hobbies': ['reading', 'writing', 'coding']
}
\end{lstlisting}

Using the dictionary you created, apply some dictionary methods on it and print the results. For example, you can do:

\begin{lstlisting}[language=Python]
d["gender"] = "female"
print(d)
\end{lstlisting}

This will output:

\begin{lstlisting}[style=output]
{
    'name': 'Alice', 
    'age': 25, 
    'hobbies': ['reading', 'writing', 'coding'], 
    'gender': 'female'
}
\end{lstlisting}

Some other dictionary methods you can try are \texttt{get}, \texttt{pop}, \texttt{keys}, \texttt{values}, \texttt{items}, \texttt{update}, \texttt{clear}, etc. You can find more information about dictionary methods here:\\

\url{https://docs.python.org/3/tutorial/datastructures.html#dictionaries}

\end{problem}

\end{document}
