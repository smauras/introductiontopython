\documentclass{homework}

\Title{Assignment 3}
\DueDate{July 7th, 2023}
\ClassName{Introduction to Programming with Python}
\ClassNumber{Columbia University Pre-College Program}
\ClassSection{Summer Session I}
\Instructor{Instructor: Daniel Kadyrov}
\Author{}

\begin{document}

\maketitle

In this assignment, you will practice using variables, floats, integers, strings, lists, and dictionary methods in Python. You will write a program that performs some simple calculations and manipulations on different types of data. You will submit your program as a .py file on the course website. The file should be named \texttt{LastName\_FirstName\_Assignment3.py} where last name and first name are replaced with your last name and first name.\\

\begin{problem}[1]

    Write a function called \texttt{is\_palindrome} that takes a string as an argument and returns \texttt{True} if the string is a palindrome (a word that is the same forward and backward) and \texttt{False} otherwise. For example, \texttt{is\_palindrome("racecar")} should return True and \texttt{is\_palindrome("python")} should return \texttt{False}. Use a loop and a conditional statement in your function.

\end{problem}

\begin{problem}[2]

    Write a class called \texttt{Rectangle} that has two attributes: \texttt{length} and \texttt{width}. The class should have a constructor method that takes two arguments and assigns them to the attributes. The class should also have two methods: \texttt{area} and \texttt{perimeter}, that return the area and perimeter of the rectangle respectively. For example, if \texttt{r = Rectangle(3, 4)}, then \texttt{r.area()} should return 12 and \texttt{r.perimeter()} should return 14.
\end{problem}

\begin{problem}[3]

    Write a function called \texttt{count\_words} that takes a list of strings as an argument and returns a dictionary that maps each word in the list to the number of times it appears. For example, \texttt{count\_words(["hello", "world", "hello", "python"])} should return \texttt{{"hello": 2, "world": 1, "python": 1}}. Use a loop and a dictionary in your function.
\end{problem}

\end{document}
