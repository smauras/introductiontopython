\documentclass{homework}

\Title{Assignment 2}
\DueDate{July 6th, 2023}
\ClassName{Introduction to Programming with Python}
\ClassNumber{Columbia University Pre-College Program}
\ClassSection{Summer Session I}
\Instructor{Instructor: Daniel Kadyrov}

\begin{document}

\maketitle

In this assignment, you will practice using variables, floats, integers, strings, lists, and dictionary methods in Python. You will write a program that performs some simple calculations and manipulations on different types of data. You will submit your program as a .py file on the course website. The file should be named \texttt{LastName\_FirstName\_Assignment2.py} where last name and first name are replaced with your last name and first name.\\

\begin{problem}[1 - For Loops]

\begin{enumerate}
\item Write a for loop that prints the numbers from 1 to 10, one per line.
\item Write a for loop that iterates over a list of names and prints “Hello, name” for each name in the list.
\item Write a for loop that calculates the sum of all the elements in a list of numbers and prints the result.
\item Write a for loop that reverses a string and prints the reversed string. For example, if the string is ``python'', the output should be ``nohtyp''.
\item Write a for loop that counts how many times each letter appears in a string and prints the letter and its frequency. For example, if the string is ``banana'', the output should be:
\begin{lstlisting}[style=output]
b: 1
a: 3
n: 2
\end{lstlisting}
\item Write a for loop that iterates over a dictionary of students and their grades and prints the name and grade of each student who passed the course. Assume that passing is 60\% or higher.
\item Write a for loop that prints the multiplication table for a given number n. For example, if n is 5, the output should be:
\begin{lstlisting}[style=output]
5 x 1 = 5
5 x 2 = 10
5 x 3 = 15
5 x 4 = 20
5 x 5 = 25
5 x 6 = 30
5 x 7 = 35
5 x 8 = 40
5 x 9 = 45
5 x 10 = 50
\end{lstlisting}
\end{enumerate}

\end{problem}

\newpage
\begin{problem}[2 - While Loops]

\begin{enumerate}
\item Write a while loop that prints the numbers from 1 to 10, one per line.
\item Write a while loop that asks the user to enter a name and prints “Hello, name” until the user enters “quit”.
\end{enumerate}
\end{problem}

\begin{problem}[3 - Contitionals]
    \begin{enumerate}
    \item Write a program that asks the user to enter a number and prints whether it is even or odd using an if else statement.
    \item Write a program that creates a list of 10 random numbers between 1 and 100 and prints the smallest and the largest number in the list using a loop and an if else statement.
    \item Write a program that creates a dictionary with the keys being the names of some fruits and the values being their prices. Then, ask the user to enter a fruit name and print the price of that fruit using an if else statement. If the fruit name is not in the dictionary, print "Sorry, we don't have that fruit."
    \end{enumerate}
\end{problem}

\end{document}
